\documentclass[t]{beamer}
\usetheme{Warsaw}
\usecolortheme{seahorse}
\usepackage{array}
\usepackage{graphicx}
\usepackage{amssymb,amsmath,mathrsfs,amsfonts}
%\usepackage[colorhighlight,display]{texpower}
%\usepackage{caption}
%\usepackage[all]{xy}
\usepackage{beamerthemesplit}
\mode<presentation>
%\usepackage{pause}
\usepackage{ulem}  % for strikethroughs
\usepackage{cancel} % for strikethroughs in math mode 
\usepackage{tikz}
\usepackage{calc}
\usetikzlibrary{shapes}
\usepackage{hyperref}
\hypersetup{pdfpagemode=FullScreen}
\usepackage{ifthen}
\usepackage{animate}
\usepackage{color}
\usepackage{type1cm}  % used for watermarking
\usepackage{eso-pic}  % used for watermarking
\usepackage[absolute,overlay]{textpos}
  \setlength{\TPHorizModule}{1mm}
  \setlength{\TPVertModule}{1mm}


\theoremstyle{plain}
\newtheorem{prop}{Proposition}
\newtheorem{thm}[prop]{Theorem}
\newtheorem{lem}[prop]{Lemma}
\newtheorem{cor}[prop]{Corollary}
\theoremstyle{definition}
\newtheorem{dfn}{Definition}
\newtheorem{rem}[prop]{Remark}
\newtheorem{ex}{Example}[section]
%\newtheorem{note}{Note}[section]
\newtheorem{exercise}{Exercise}[section]
\newcommand{\nin}{\noindent}
\newcommand{\ds}{\displaystyle}
\renewcommand{\figurename}{Figure \arabic{figure}}
\renewcommand*\familydefault{\sfdefault} 


%%%%%%%%%%%%%%%%%%%%%%%%%%5
%%%%%%%%%%%%%%%%%%%%%%%%%%%%
%%%% some commands that have different meaning in the article/presentation modes

\newcommand{\vvfill}{\mode<presentation>{\vfill}  \mode<article>{\medskip}}   %vfill in presentation only
\newcommand{\sketchspace}{ 
\mode<article>{ \medskip\noindent{\textbf{Sketch:}} \vspace*{6cm} }
\mode<presentation>{ } 
}
\newcommand{\examplespace}{ 
\mode<article>{ \medskip\noindent{\textbf{Example:}} \vspace{6cm} }
\mode<presentation>{ } 
}
\newcommand{\artsmspace}{\mode<article>{\vspace*{2cm}} }  %small space in article mode
\newcommand{\artlargespace}{\mode<article>{\vspace*{6cm}} }  %large space in article mode

\newcommand{\dx}{\,dx}

\newcommand{\soln}{{\textbf{Solution: }}\,\,\,}
\newcommand{\disp}{\displaystyle}

\newcommand{\makedate}{\vvfill
\begin{picture}(10,10)  
\put(260,-20){\mbox{\tiny{\today}}}
\end{picture}
}

\newcommand{\pd}[2]{\dfrac{\partial#1}{\partial#2}}
\newcommand{\pD}[2]{\dfrac{\partial^2#1}{\partial#2^2}}
\newcommand{\pdd}[3]{\dfrac{\partial^2#1}{\partial#2 \partial#3}}


\normalem %stops the ulem package making all the emphs into underlines...
 
 
 
 \newcommand{\refandrev}[2]{
 \begin{small}
  \hspace{6cm}
  \begin{minipage}[r]{8cm}
  Stewart,    Chapter #1   \\
  Review:  \parbox[t]{6cm}{#2}
\end{minipage}
\end{small}
}



\newcounter{heading}
\setcounter{section}{1}
\setcounter{heading}{0}

\newcommand{\makeheading}[1]{\medskip\begin{large}\noindent\textbf{{#1}}\end{large}\smallskip}

%\newenvironment{head}[1]{\medskip\stepcounter{heading}\noindent\textbf{\hspace{0.2cm}{#1}.}}{}
\newcommand{\newhead}[1]{\medskip\stepcounter{heading}\noindent\textbf{\hspace{0.2cm}{#1}.}}


\newcommand{\pf}[1]{\noindent\textit{Proof.}\vspace*{#1 cm}}
\newcommand{\sol}[1]{\noindent\textit{Solution.}\vspace*{#1 cm}}
\newcommand{\further}[1]{\begin{small}\noindent\textit{Further reading: #1}\end{small}}
\newcommand{\exr}[1]{\begin{footnotesize}\noindent\textit{\textbf{Exercises:} Stewart #1}\end{footnotesize}}


% Sets of numbers
\newcommand{\C}{\mathbb{C}}
\newcommand{\RR}{\mathbb{R}}
\newcommand{\Z}{\mathbb{Z}}
\newcommand{\N}{\mathbb{N}}
\newcommand{\Q}{\mathbb{Q}}

% Partitions
\newcommand{\PP}{\mathcal{P}}

% Limits
\newcommand{\limm}[1]{\displaystyle \lim_{n\to #1}}

% Backslash
\newcommand{\bs}{\backslash}

% functions
\newcommand{\cosec}{\mathrm{cosec}}
\newcommand{\cosech}{\mathrm{cosech}}
\newcommand{\sech}{\mathrm{sech}}
\newcommand{\Li}{\mathrm{Li}}
\newcommand{\si}{\mathrm{Si}}
\newcommand{\erf}{\mathrm{erf}}

% Domain and Range
\newcommand{\Dom}{\mathrm{Dom}}
\newcommand{\Codom}{\mathrm{Codom}}
\newcommand{\Range}{\mathrm{Ran}}



\title{Week 13: Series}

\begin{document}

\frame{\titlepage}

\setcounter{tocdepth}{2}
\frame{\tableofcontents

}

\AtBeginSection[]
{
\begin{frame}<beamer> 
\tableofcontents[currentsection]  % show TOC and highlight current section
\end{frame}
}

\section{Geometric Series}

\begin{frame}
\footnotesize
\frametitle{Series}

\textbf{Definition}: Given a series $\displaystyle\sum_{n=1}^{\infty} a_n = a_1 + a_2 + a_3 + \cdots$,  let $s_n$ denote its $n$th partial sum:

$$s_n = \sum_{i=1}^{n}a_i = a_1 + a_2 + \cdots + a_n$$

If the sequence $\{s_n\}$ is convergent and $\limm{\infty} s_n = s$ exists as a real number, then the series $\sum a_n$ is called \textbf{convergent} and we write

$$a_1 + a_2 + \cdots + a_n + \cdots = s \text{  or  } \sum_{n=1}^{\infty} a_n = s$$

The number $s$ is called the \textbf{sum} of the series.  If the sequence $\{s_n\}$ is divergent, then the series is called \textbf{divergent}.

\medskip

Thus $\displaystyle\sum_{n=1}^{\infty} a_n = \limm{\infty} \sum_{i=1}^{n} a_i$

\end{frame}

\begin{frame}
\footnotesize
\frametitle{Geometric series}

An important example of an infinite series is the \textbf{geometric series}.

$$a + ar + ar^2 + ar^3 + \cdots + ar^{n-1} + \cdots = \sum_{n=1}^{\infty}ar^{n-1}  \qquad a \neq 0$$

\begin{itemize}
	\item If $r = 1$, the limit diverges to infinity because it becomes $n \cdot a$
	\item If $r \neq 1$, then 
\end{itemize}

\begin{align*}
	s_n &= a + ar + ar^2 + \cdots + ar^{n-1} \\
	rs_n &= ar + ar^2 + \cdots + ar^{n-1} + ar^n && \text{multiply both sides by r} \\
	s_n - rs_n &= a - ar^n && \text{substracting these two equations} \\
	s_n &= \frac{a(1-r^n)}{1 - r}
\end{align*}

\begin{itemize}
	\item If $-1 < r < 1$ (same as $|r| < 1$,)   then $\limm{\infty} s_n = \frac{a}{1-r}$ since $r^n \rightarrow 0$
	\item If $r \leq 1$ or $r > 1$ (same as $|r| \geq 1$), then $\{ r^n \}$ is divergent since $r^n \rightarrow $ DNE
\end{itemize}

\end{frame}


\begin{frame}
\frametitle{Example}

Is the series $\displaystyle\sum_{n=1}^{\infty}2^{2n}3^{1-n}$ convergent? \pause

\medskip

Let's rewrite the $n$th term of the series in the form $ar^{n-1}$:

$$\sum_{n=1}^{\infty} 2^{2n}3^{1-n} = \sum_{n=1}^{\infty}(2^2)^n 3^{-(n-1)} = \sum_{n=1}^{\infty} \frac{4^n}{3^{n-1}} = \sum_{n=1}^{\infty} 4 \frac{4}{3}^{n-1}$$

We recognize this series as a geometric series with $a = 4$ and $r = \frac{4}{3}$.    Since $r > 1$, the
series diverges.

\end{frame}

\begin{frame}
\frametitle{Example}
A drug is administered to a patient at the same time every day.   Suppose the concentration of the drug is $C_n$ (measured in mg/mL) after the injection on the $n$th day.   Before the injection the next day,  only 3$0\%$ of the drug remains in the bloodstream and the daily dose raises the concentration by 0.2 mg/mL.  (a) Find the concentration after three days, (b) what is the concentration after the $n$th does? (c) what is the limiting concentration? \pause

\medskip

(a). The concentration after the next day is 

\begin{align*}
	C_{n+1} &= 0.2 + 0.3C_n\\
	C_1 &= 0.2 + 0.3C_0 = 0.2\\
	C_2 &= 0.2 + 0.3(C_1=0.2) = 0.26\\
	C_3 &= 0.2 + 0.3(C_2=0.26) = 0.278\\
\end{align*}

\textit{(continued next slide)}

\end{frame}

\begin{frame}
\textit{continued...}

(b) After the $n$th does the concentration is

$$C_n = 0.2 + 0.2(0.3) + 0.2(0.3)^2 + \cdots + 0.2(0.3)^{n-1}$$

Given $a = 0.2$ and $r = 0.3$, so we have

$$C_n = \frac{a(1 - r^n)}{1 - r} = \frac{0.2[1 - 0.3^n]}{1 - 0.3}=\frac{2}{7}[1 - (0.3)^n] \text{ mg/mL }$$

(c) Since $0.3 < 1$, thus $r^n \rightarrow 0$, thus

$$\limm{\infty} C_n = \limm{\infty}\frac{2}{7}(1 - 0) = \frac{2}{7} \text{ mg/mL }$$


\end{frame}

\begin{frame}
\footnotesize
\frametitle{Exercise}

Determine whether the geometric series is convergent or divergent.  If it is convergent, find its sum.

\begin{itemize}
	\item $3 - 4 + \frac{16}{3} - \frac{64}{9}$ %stewart 11.2 No. 17
		\begin{itemize}
			\item Diverges
		\end{itemize}
	\item $4 + 3 + \frac{9}{3} + \frac{27}{16}$ %stewart 11.2 No. 18
		\begin{itemize}
			\item Converges at 16
		\end{itemize}
	\item $\displaystyle\sum_{n=1}^{\infty}12(0.73)^{n-1}$ %stewart 11.2 No. 21
		\begin{itemize}
			\item Converges at $\frac{400}{9}$
		\end{itemize}
	\item $\displaystyle\sum_{n=1}^{\infty}\frac{5}{\pi^n}$ %stewart 11.2 No. 22
		\begin{itemize}
			\item Converges at $\frac{5}{\pi - 1}$
		\end{itemize}
	\item $\displaystyle\sum_{n=1}^{\infty}\frac{6 \cdot 2^{2n-1}}{3^n}$ %stewart 11.2 No. 26
	\begin{itemize}
		\item Diverges
	\end{itemize}
\end{itemize}


\end{frame}


\section{Power Series}

\begin{frame}

\footnotesize

\frametitle{Power Series}

A \textbf{power series} is a series of the form

$$\sum_{n=0}^{\infty} c_nx^n = c_0 + c_1x + c_2x^2 + c_3x^3 + \cdots$$

where $x$ is a variable and the $c_n$'s are constants called the coefficients of the series.   For each fixed $x$, the series is a series of constants that we can test for convergence or divergence.

\medskip

A power series may converge for some values of $x$ and diverge for other values of $x$.  For example, if we take $c_n = 1$ for all $n$, the power series becomes the geometric series

$$\sum_{n=0}^{\infty}x^n = 1 + x + x^2 + \cdots + x^n + \cdots $$

which converges when $-1 < x < 1$ and diverges when $|x| \geq 1$

A \textbf{power series centered at} $a$ can be written in the form:

$$\sum_{n=0}^{\infty} c_n(x-a)^n = c_0 + c_1(x - a) + c_2 (x - a)^2 + \cdots$$

\end{frame}

\begin{frame}

\frametitle{Ratio Test}

\begin{itemize}
	\item If $\limm{\infty} \displaystyle\left|\frac{a_{n+1}}{a_n}\right| = L < 1$, then the series $\displaystyle\sum_{n=1}^{\infty} a_n$ is convergent
	\item If $\limm{\infty} \displaystyle\left|\frac{a_{n+1}}{a_n}\right| = L > 1$, then the series $\displaystyle\sum_{n=1}^{\infty} a_n$ is divergent
	\item If $\limm{\infty} \displaystyle\left|\frac{a_{n+1}}{a_n}\right| =  1$, then the series $\displaystyle\sum_{n=1}^{\infty} a_n$ is inconclusive
\end{itemize}


\end{frame}

\begin{frame}

\frametitle{Example}

Is $\displaystyle\sum_{n=1}^{\infty}(-1)^n \frac{n^3}{3^n}$ convergent? \pause

\begin{align*}
\displaystyle\left|\frac{a_{n+1}}{a_n}\right|  &= \left|  \frac{   \frac{(-1)^{n+1}(n + 1)^3}{3^{n + 1}}}{(-1)^n \frac{n^3}{3^n}}  \right| = \frac{(n + 1)^3}{3^{n + 1}} \cdot \frac{3^n}{n^3}\\
&= \frac{1}{3}\left( \frac{n + 1}{n}\right)^3 = \frac{1}{3}\left( 1 + \frac{1}{n} \right)^3 \rightarrow \frac{1}{3} < 1
\end{align*}

Thus, by the Ratio Test, the given series is convergent.

\end{frame}

\begin{frame}
\frametitle{Example}

Let's try it on power series.  Is the series $\displaystyle\sum_{n=0}^{\infty} n!x^n$ convergent? \pause

$$\limm{\infty}\left|\frac{a_{n+1}}{a_n}\right| = \limm{\infty}\left| \frac{(n + 1)!x^{n+1}}{n!x^n}\right| = \limm{\infty}(n + 1) |x| = \infty$$

Thus, the series diverges when $x \neq 0$ and converges only when $x = 0$

\end{frame}

\begin{frame}

\frametitle{Example}

For what values of $x$ does the series $\displaystyle\sum_{n=1}^{\infty}\frac{(x-3)^n}{n}$ converge? \pause

\begin{align*}
\limm{\infty}\left|\frac{a_{n+1}}{a_n}\right| &= \left| \frac{(x-3)^{n + 1}}{n + 1} \cdot \frac{n}{(x - 3)^n}  \right|\\
&= \frac{1}{1 + \frac{1}{n}} |x - 3| \rightarrow |x - 3| \qquad \text{ as } n \rightarrow \infty
\end{align*}

By the Ratio Test, the given series is convergent when $|x - 3| < 1$ and divergent when $|x - 3| > 1$.  Now $|x - 3| < 1 \Leftrightarrow -1 < x - 3 < 1 \Leftrightarrow 2 < x < 4$,  thus the series converges when $2 < x < 4$ and diverges when $x < 2$ or $x > 4$.  

\end{frame}

\begin{frame}

\frametitle{Radius of convergence}

For a given power series $\sum_{n=0}^{\infty} c_n (x - a)^n$, there are only three possibilities:

\begin{itemize}
	\item The series converges only when $x = a$
	\item The series converges for all $x$
	\item There is a positive number $R$ such that the series converges if $|x - a| < R$ and diverges if $|x - a| > R$
\end{itemize}

The number $R$ is called \textbf{radius of convergence}. 

\end{frame}

\begin{frame}

\frametitle{Example}

Find the radius of convergence of $\sum_{n=0}^{\infty} \frac{(-3)^nx^n}{\sqrt{n + 1}}$ \pause

\begin{align*}
\limm{\infty}\left|\frac{a_{n+1}}{a_n}\right| &=  \left| \frac{(-3)^{n + 1} x^{n + 1}}{\sqrt{n + 2}} \cdot \frac{\sqrt{n + 1}}{(-3)^n x^n}\right| = \left| -3x \sqrt{\frac{n + 1}{n + 2}}\right|\\
&= 3 \sqrt{\frac{1 + (1/n)}{1 + (2/n)}} |x| \rightarrow 3 |x| \qquad \text{ as } n \rightarrow \infty
\end{align*}

By the Ratio Test, the given series converges if $3 |x| < 1$ and diverges if $3 |x| > 1$.  Thus it converges if $|x| < \frac{1}{3}$ and diverges if $|x| > \frac{1}{3}$.  This means that the radius of convergence is $R = \frac{1}{3}$.

\end{frame}

\begin{frame}

\frametitle{Example}

Find the radius of convergence of $\displaystyle\sum_{n=0}^{\infty} \frac{n(x + 2)^n}{3^{n + 1}}$ \pause

\begin{align*}
\limm{\infty}\left|\frac{a_{n+1}}{a_n}\right| &= \left| \frac{(n + 1)(x + 2)^{n + 1}}{3^{n + 2}} \cdot \frac{3^{n + 1}}{n(x + 2)^n}\right|\\
&= \left(  1 + \frac{1}{n}  \right) \frac{|x + 2|}{3} \rightarrow \frac{|x + 2|}{3} \qquad \text{ as } n \rightarrow \infty
\end{align*}

By the Ratio Test, the given series converges if $\frac{|x + 2|}{3}< 1$ and diverges if $\frac{|x + 2|}{3} > 1$.  Thus it converges if $|x + 2| < 3$ and diverges if $|x + 2| > 3$.  This means that the radius of convergence is $R = 3$.

\end{frame}

\begin{frame}

\frametitle{Exercise}

Find the radius of convergence.

\begin{itemize}
	\item $\displaystyle\sum_{n=1}^{\infty} (-1)^n nx^n$  %stewart 11.8 No. 3
	\begin{itemize}
		\item Ans: 1
	\end{itemize}
	\item $\displaystyle\sum_{n=1}^{\infty} \frac{x^n}{2n-1}$  %stewart 11.8 No. 5
	\begin{itemize}
		\item Ans: 1
	\end{itemize}
	\item $\displaystyle\sum_{n=1}^{\infty} \frac{(-1)^n 4^n}{\sqrt{n}}x^n$  %stewart 11.8 No. 11
	\begin{itemize}
		\item Ans: $\frac{1}{4}$
	\end{itemize}
	\item $\displaystyle\sum_{n=0}^{\infty} \frac{(x - 2)^n}{n^2 + 1}$  %stewart 11.8 No. 15
	\begin{itemize}
		\item Ans: 1
	\end{itemize}

\end{itemize}

\end{frame}


\end{document}