\documentclass[t]{beamer}
\usetheme{Warsaw}
\usecolortheme{seahorse}
\usepackage{array}
\usepackage{graphicx}
\usepackage{amssymb,amsmath,mathrsfs,amsfonts}
%\usepackage[colorhighlight,display]{texpower}
%\usepackage{caption}
%\usepackage[all]{xy}
\usepackage{beamerthemesplit}
\mode<presentation>
%\usepackage{pause}
\usepackage{ulem}  % for strikethroughs
\usepackage{cancel} % for strikethroughs in math mode 
\usepackage{tikz}
\usepackage{calc}
\usetikzlibrary{shapes}
\usepackage{hyperref}
\hypersetup{pdfpagemode=FullScreen}
\usepackage{ifthen}
\usepackage{animate}
\usepackage{color}
\usepackage{type1cm}  % used for watermarking
\usepackage{eso-pic}  % used for watermarking


\theoremstyle{plain}
\newtheorem{prop}{Proposition}
\newtheorem{thm}[prop]{Theorem}
\newtheorem{lem}[prop]{Lemma}
\newtheorem{cor}[prop]{Corollary}
\theoremstyle{definition}
\newtheorem{dfn}{Definition}
\newtheorem{rem}[prop]{Remark}
\newtheorem{ex}{Example}[section]
%\newtheorem{note}{Note}[section]
\newtheorem{exercise}{Exercise}[section]
\newcommand{\nin}{\noindent}
\newcommand{\ds}{\displaystyle}
\renewcommand{\figurename}{Figure \arabic{figure}}



\renewcommand*\familydefault{\sfdefault} 




%%%%%%%%%%%%%%%%%%%%%%%%%%5
%%%%%%%%%%%%%%%%%%%%%%%%%%%%
%%%% some commands that have different meaning in the article/presentation modes

\newcommand{\vvfill}{\mode<presentation>{\vfill}  \mode<article>{\medskip}}   %vfill in presentation only
\newcommand{\sketchspace}{ 
\mode<article>{ \medskip\noindent{\textbf{Sketch:}} \vspace*{6cm} }
\mode<presentation>{ } 
}
\newcommand{\examplespace}{ 
\mode<article>{ \medskip\noindent{\textbf{Example:}} \vspace{6cm} }
\mode<presentation>{ } 
}
\newcommand{\artsmspace}{\mode<article>{\vspace*{2cm}} }  %small space in article mode
\newcommand{\artlargespace}{\mode<article>{\vspace*{6cm}} }  %large space in article mode

\newcommand{\dx}{\,dx}

\newcommand{\soln}{{\textbf{Solution: }}\,\,\,}
\newcommand{\disp}{\displaystyle}

\newcommand{\makedate}{\vvfill
\begin{picture}(10,10)  
\put(260,-20){\mbox{\tiny{\today}}}
\end{picture}
}

\newcommand{\pd}[2]{\dfrac{\partial#1}{\partial#2}}
\newcommand{\pD}[2]{\dfrac{\partial^2#1}{\partial#2^2}}
\newcommand{\pdd}[3]{\dfrac{\partial^2#1}{\partial#2 \partial#3}}


\normalem %stops the ulem package making all the emphs into underlines...
 
 
 
 \newcommand{\refandrev}[2]{
 \begin{small}
  \hspace{6cm}
  \begin{minipage}[r]{8cm}
  Stewart,    Chapter #1   \\
  Review:  \parbox[t]{6cm}{#2}
\end{minipage}
\end{small}
}



\newcounter{heading}
\setcounter{section}{1}
\setcounter{heading}{0}

\newcommand{\makeheading}[1]{\medskip\begin{large}\noindent\textbf{{#1}}\end{large}\smallskip}

%\newenvironment{head}[1]{\medskip\stepcounter{heading}\noindent\textbf{\hspace{0.2cm}{#1}.}}{}
\newcommand{\newhead}[1]{\medskip\stepcounter{heading}\noindent\textbf{\hspace{0.2cm}{#1}.}}


\newcommand{\pf}[1]{\noindent\textit{Proof.}\vspace*{#1 cm}}
\newcommand{\sol}[1]{\noindent\textit{Solution.}\vspace*{#1 cm}}
\newcommand{\further}[1]{\begin{small}\noindent\textit{Further reading: #1}\end{small}}
\newcommand{\exr}[1]{\begin{footnotesize}\noindent\textit{\textbf{Exercises:} Stewart #1}\end{footnotesize}}


% Sets of numbers
\newcommand{\C}{\mathbb{C}}
\newcommand{\RR}{\mathbb{R}}
\newcommand{\Z}{\mathbb{Z}}
\newcommand{\N}{\mathbb{N}}
\newcommand{\Q}{\mathbb{Q}}

% Partitions
\newcommand{\PP}{\mathcal{P}}

% Limits
\newcommand{\limm}[1]{\displaystyle \lim_{x\to #1}}

% Backslash
\newcommand{\bs}{\backslash}

% functions
\newcommand{\cosec}{\mathrm{cosec}}
\newcommand{\cosech}{\mathrm{cosech}}
\newcommand{\sech}{\mathrm{sech}}
\newcommand{\Li}{\mathrm{Li}}
\newcommand{\si}{\mathrm{Si}}
\newcommand{\erf}{\mathrm{erf}}

% Domain and Range
\newcommand{\Dom}{\mathrm{Dom}}
\newcommand{\Codom}{\mathrm{Codom}}
\newcommand{\Range}{\mathrm{Ran}}



\title{Week 9:  Integration Techniques Part 1}
%\date{October 2 -- October 5, 2012}

\begin{document}

\frame{\titlepage}

\setcounter{tocdepth}{2}
\frame{\tableofcontents
}

\AtBeginSection[]
{
\begin{frame}<beamer> 
\tableofcontents[currentsection]  % show TOC and highlight current section
\end{frame}
}


\section{Integration by parts}

\begin{frame}
\frametitle{Integration by parts}

Let's start with the product rule:
\begin{align*}
(f(x)g(x))' &= f'(x)g(x) + f(x)g'(x)\\
 f(x)g'(x)  & = (f(x)g(x))'  - f'(x)g(x)\\
\end{align*}

\vspace{-1em}

Integrating both sides, we get:
\begin{align*}
\displaystyle\int f(x)g'(x)\,dx &= f(x)g(x) - \int g(x)f'(x)\,dx
\end{align*}

If we let $u = f(x)$ and $v = g(x)$, so $du=f'(x)\,dx$ and $dv = g'(x)\,dx$, we obtain an equivalent (perhaps easier to memorize) formula:
\[\displaystyle\int u\, dv = uv - \int v \, du.\]

\end{frame}

\begin{frame}
\frametitle{Example} 

Evaluate $ \int x e^{x} \, dx.$ \pause

\medskip

Let $u=x$,  $dv = e^x \,dx$,  $du = dx$,  and $v = e^x$

\begin{align*}
\int u\, dv &= uv - \int v \, du \\
\int xe^x \,dx &= xe^x - \int e^x \,dx\\
                       &= xe^x - e^x + C
\end{align*}

\end{frame}

\begin{frame}
\frametitle{Example} 

Evaluate $ \int x \sin{x} \, dx.$ \pause

\medskip

Let $u=x$,  $dv = \sin{x} \,dx$,  $du = dx$,  and $v = -\cos{x}$

\begin{align*}
\int u\, dv &= uv - \int v \, du \\
\int xe^x \,dx &= x(-\cos{x})- \int (-\cos{x}) \,dx\\
                       &= -x\cos{x} + \sin{x} + C
\end{align*}

\end{frame}

\begin{frame}
\frametitle{Exercise} 

\begin{enumerate}
	\item $\int \ln{x} \,dx$
	\begin{itemize}
		\item Ans: $x\ln{x} -x + C$ %Example 7.1.2
	\end{itemize}
	\item $\int t^2e^t \,dt$
	\begin{itemize}
		\item Ans: $t^2e^t - 2te^t + 2e^t + C$ %Example 7.1.3
	\end{itemize}
	\item $\int (x^2 + 2x)\cos x \dx$ %7.1.3 No. 7
	\begin{itemize}
		\item $(x^2 + 2x)\sin{x} + (2x + 2)\cos{x} - 2\sin{x} + C$
	\end{itemize}
\end{enumerate}

\end{frame}


\section{Trigonometric integrals ($\sin^m{x}\cos^n{x})$}

%\begin{frame}
%\frametitle{Trigonometric integrals}
%%%%%%%%%%%%%%%%%%%%%%%%%%%%%%%%%%%%%%%%
%
%\noindent We examine integrals of the form
%\begin{enumerate}
%\item $\ds\int \sin^mx \cos^nx\,dx$
%\item $\ds\int \tan^mx \sec^nx\,dx$
%\end{enumerate}
%
%\vspace*{.5cm}
%
%\noindent There are some strategies to evaluating these integrals, but occasionally we must use some tricks and ingenuity.
%
%\noindent We begin by looking at integrals involving powers of $\sin x$ and $\cos x$.
%\end{frame}

\begin{frame}
\footnotesize

\[ \ds\int \sin^mx \cos^nx\,dx.\]
There are two cases to consider.  The first case is if \textbf{at least one} of $m$ or $n$ is odd, and the second case is if \textbf{both} $m$ and $n$ are even.

\newhead{Case 1: Integrals with an odd power of $\sin x$ \emph{or} an odd power of $\cos x$}
\begin{itemize}
\item If there is an odd power of $\sin x$ (i.e. if $m$ above is odd), we save one factor of $\sin x$ and use $\sin^2x=1-\cos^2x$ to express the remaining factors in terms of $\cos x$. Then substitute $u=\cos x$.
\item If there is an odd power of $\cos x$ (i.e. if $n$ above is odd), we save one factor of $\cos x$ and use $\cos^2x=1-\sin^2x$ to express the remaining factors in terms of $\sin x$. Then substitute $u=\sin x$.
\item If the powers of both $\sin x$ and $\cos x$ are odd, we may use either of the above strategies
\end{itemize}
\end{frame}

\begin{frame}
\frametitle{Example} 

Evaluate $\ds\int\cos^3x\,dx$.  \pause

\begin{itemize}
	\item $\ds\int\cos^3x\,dx = \ds\int\cos^2x \cdot \cos{x}\,dx = \ds\int 1 - \sin^2x \cdot \cos{x} \,dx$
	\item Let $u = \sin{x}$ and $du = \cos{x} \,dx$
	\item $\ds\int (1 - u^2)\,du = u - \frac{1}{3}u^3 + C = \sin{x} - \frac{1}{3}\sin^3{x} + C$
\end{itemize}

\end{frame}

\begin{frame}
\frametitle{Example} 

Evaluate $\ds\int\sin^5x\cos^2x\,dx$. \pause

\begin{itemize}
	\item $\ds\int\sin^5x\cos^2x\,dx = \ds\int(\sin^2x)^2\cos^2x\sin{x}\,dx = (1-\cos^2{x})^2 \cos^2{x} \sin{x}\,dx$
	\item Let $u = \cos{x}$ and $du = -\sin{x} \,dx$
	\item $\ds\int (1 - u^2)^2 u^2 (-\,du) = -\ds\int (u^2 - 2u^4 + u^6) \,du = -\left(\dfrac{u^3}{3} - 2\dfrac{u^5}{5} + \dfrac{u^7}{7}\right) + C = -\dfrac{1}{3}\cos^3{x} + \dfrac{2}{5}\cos^5{x} - \dfrac{1}{7}\cos^7{x} + C$
\end{itemize}

\end{frame}

\begin{frame}
\frametitle{Exercise} 

Evaluate $\ds\int\sin^2x\cos^3x\,dx$. \pause

\begin{itemize}
	\item $\ds\int\sin^2x\cos^3x\,dx = \ds\int\sin^2x\cos^2x \cos{x}\,dx = \ds\int\sin^2x (1 - \sin^2x) \cos{x}\,dx$.
	\item Let $u = \sin{x}$ and $du = \cos{x} \,dx$
	\item $\ds\int u^2 (1-u^2) \,du = \ds\int(u^2 - u^4) \, du = \dfrac{1}{3}u^3 - \dfrac{1}{5}u^5 + C = \dfrac{1}{3}\sin^3{x} - \dfrac{1}{5}\sin^5{x} + C$
\end{itemize}

\end{frame}

\begin{frame}
\newhead{Case 2: Integrals where both powers of $\sin x$ and $\cos x$ are even}

We use the half-angle identities
\begin{align*}
\sin^2x&=\tfrac{1}{2}(1-\cos2x)\\
\cos^2x&=\tfrac{1}{2}(1+\cos2x)\\ 
%\intertext{and progressively lower the powers until the integral can be evaluated. Sometimes the identity
%\sin x\cos x&=\tfrac{1}{2}\sin2x}
\end{align*}

It is also useful to use the identity:

$$\sin{x}\cos{x} = \dfrac{1}{2}\sin{2x}$$

\end{frame}

\begin{frame}
\frametitle{Example} 

Evaluate $\ds\int\sin^2x\,dx$.  \pause

\begin{align*}
\ds\int\sin^2x\,dx &= \dfrac{1}{2} \int (1 - \cos{2x}) \,dx\\
                               &= \dfrac{1}{2} (x - \dfrac{1}{2}\sin{2x})
\end{align*}

\end{frame}

\begin{frame}
\frametitle{Example} 

\footnotesize

Evaluate $\ds\int\sin^2t \cos^4{t} \,dt$.  \pause

\begin{align*}
\ds\int\sin^2t \cos^4{t} \,dt &= \dfrac{1}{4} \int (4 \sin^2{t} \cos^2{t}) \cos^2{t} \,dt  \\
                                               &= \dfrac{1}{4} \int (2 \sin{t} \cos{t})^2 \dfrac{1}{2}(1 + \cos{2t}) \,dt && \text{[half-angle identity]}\\
                                               &= \dfrac{1}{8} \int (\sin{2t})^2 (1 + \cos{2t}) \,dt &&\text{[identity]}\\
                                               &= \dfrac{1}{8} \int (\sin^2{2t} + \sin^2{2t}\cos{2t})\,dt \\
	                                           &= \dfrac{1}{8} \int \sin^2{2t}\,dt + \dfrac{1}{8} \int  \sin^2{2t}\cos{2t}\,dt \\
	                                           &= \dfrac{1}{8} \int \dfrac{1}{2}(1 - \cos{4t})\,dt + \dfrac{1}{8} \left(\dfrac{\sin^3{2t}}{6} \right) && \text{[Sub. u = $\sin{2t}$]}\\
	                                           &= \dfrac{1}{16} \left(t - \dfrac{1}{4}\sin{4t}\right) + \dfrac{1}{8}\left(\dfrac{\sin^3{2t}}{6} \right) \\
\end{align*}

\end{frame}

\begin{frame}
\frametitle{Exercise} 

\footnotesize

Evaluate $\ds\int\cos^4{2t} \,dt$.  \pause

\begin{align*}
\ds\int\cos^4{2t} \,dt &=  \ds\int (\cos^2{2t})^2\\
 									  &= \ds\int \left(\dfrac{1}{2}(1 + \cos{4t})\right)^2 && \text{[half-angle identity}] \\
 									  &= \ds\int \dfrac{1}{4}(1 + 2 \cos{4t} + \cos^2{4t})\,dt\\
 									 &= \dfrac{1}{4}\ds\int (1 + 2 \cos{4t} + \dfrac{1}{2}(1 + \cos{8t}))\,dt && \text{[half-angle identity}] \\
 									 &= \dfrac{1}{4}\ds\int (\dfrac{3}{2} + 2 \cos{4t} + \dfrac{1}{2}\cos{8t})\,dt \\
 									 &= \dfrac{1}{4}\left(\dfrac{3}{2}t + \dfrac{1}{2}\sin{4t} + \dfrac{1}{16}\sin{8t}\right)
\end{align*}

\end{frame}





\end{document}